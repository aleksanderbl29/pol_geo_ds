% Options for packages loaded elsewhere
\PassOptionsToPackage{unicode}{hyperref}
\PassOptionsToPackage{hyphens}{url}
%
\documentclass[
  a4paper,
  DIV=11,
  numbers=noendperiod]{scrartcl}

\usepackage{amsmath,amssymb}
\usepackage{setspace}
\usepackage{iftex}
\ifPDFTeX
  \usepackage[T1]{fontenc}
  \usepackage[utf8]{inputenc}
  \usepackage{textcomp} % provide euro and other symbols
\else % if luatex or xetex
  \usepackage{unicode-math}
  \defaultfontfeatures{Scale=MatchLowercase}
  \defaultfontfeatures[\rmfamily]{Ligatures=TeX,Scale=1}
\fi
\usepackage{lmodern}
\ifPDFTeX\else  
    % xetex/luatex font selection
    \setmainfont[]{Times New Roman}
    \setsansfont[]{Times New Roman}
\fi
% Use upquote if available, for straight quotes in verbatim environments
\IfFileExists{upquote.sty}{\usepackage{upquote}}{}
\IfFileExists{microtype.sty}{% use microtype if available
  \usepackage[]{microtype}
  \UseMicrotypeSet[protrusion]{basicmath} % disable protrusion for tt fonts
}{}
\makeatletter
\@ifundefined{KOMAClassName}{% if non-KOMA class
  \IfFileExists{parskip.sty}{%
    \usepackage{parskip}
  }{% else
    \setlength{\parindent}{0pt}
    \setlength{\parskip}{6pt plus 2pt minus 1pt}}
}{% if KOMA class
  \KOMAoptions{parskip=half}}
\makeatother
\usepackage{xcolor}
\usepackage[top=2cm,bottom=4cm,left=2cm,right=2cm,heightrounded]{geometry}
\setlength{\emergencystretch}{3em} % prevent overfull lines
\setcounter{secnumdepth}{5}
% Make \paragraph and \subparagraph free-standing
\makeatletter
\ifx\paragraph\undefined\else
  \let\oldparagraph\paragraph
  \renewcommand{\paragraph}{
    \@ifstar
      \xxxParagraphStar
      \xxxParagraphNoStar
  }
  \newcommand{\xxxParagraphStar}[1]{\oldparagraph*{#1}\mbox{}}
  \newcommand{\xxxParagraphNoStar}[1]{\oldparagraph{#1}\mbox{}}
\fi
\ifx\subparagraph\undefined\else
  \let\oldsubparagraph\subparagraph
  \renewcommand{\subparagraph}{
    \@ifstar
      \xxxSubParagraphStar
      \xxxSubParagraphNoStar
  }
  \newcommand{\xxxSubParagraphStar}[1]{\oldsubparagraph*{#1}\mbox{}}
  \newcommand{\xxxSubParagraphNoStar}[1]{\oldsubparagraph{#1}\mbox{}}
\fi
\makeatother


\providecommand{\tightlist}{%
  \setlength{\itemsep}{0pt}\setlength{\parskip}{0pt}}\usepackage{longtable,booktabs,array}
\usepackage{calc} % for calculating minipage widths
% Correct order of tables after \paragraph or \subparagraph
\usepackage{etoolbox}
\makeatletter
\patchcmd\longtable{\par}{\if@noskipsec\mbox{}\fi\par}{}{}
\makeatother
% Allow footnotes in longtable head/foot
\IfFileExists{footnotehyper.sty}{\usepackage{footnotehyper}}{\usepackage{footnote}}
\makesavenoteenv{longtable}
\usepackage{graphicx}
\makeatletter
\def\maxwidth{\ifdim\Gin@nat@width>\linewidth\linewidth\else\Gin@nat@width\fi}
\def\maxheight{\ifdim\Gin@nat@height>\textheight\textheight\else\Gin@nat@height\fi}
\makeatother
% Scale images if necessary, so that they will not overflow the page
% margins by default, and it is still possible to overwrite the defaults
% using explicit options in \includegraphics[width, height, ...]{}
\setkeys{Gin}{width=\maxwidth,height=\maxheight,keepaspectratio}
% Set default figure placement to htbp
\makeatletter
\def\fps@figure{htbp}
\makeatother
% definitions for citeproc citations
\NewDocumentCommand\citeproctext{}{}
\NewDocumentCommand\citeproc{mm}{%
  \begingroup\def\citeproctext{#2}\cite{#1}\endgroup}
\makeatletter
 % allow citations to break across lines
 \let\@cite@ofmt\@firstofone
 % avoid brackets around text for \cite:
 \def\@biblabel#1{}
 \def\@cite#1#2{{#1\if@tempswa , #2\fi}}
\makeatother
\newlength{\cslhangindent}
\setlength{\cslhangindent}{1.5em}
\newlength{\csllabelwidth}
\setlength{\csllabelwidth}{3em}
\newenvironment{CSLReferences}[2] % #1 hanging-indent, #2 entry-spacing
 {\begin{list}{}{%
  \setlength{\itemindent}{0pt}
  \setlength{\leftmargin}{0pt}
  \setlength{\parsep}{0pt}
  % turn on hanging indent if param 1 is 1
  \ifodd #1
   \setlength{\leftmargin}{\cslhangindent}
   \setlength{\itemindent}{-1\cslhangindent}
  \fi
  % set entry spacing
  \setlength{\itemsep}{#2\baselineskip}}}
 {\end{list}}
\usepackage{calc}
\newcommand{\CSLBlock}[1]{\hfill\break\parbox[t]{\linewidth}{\strut\ignorespaces#1\strut}}
\newcommand{\CSLLeftMargin}[1]{\parbox[t]{\csllabelwidth}{\strut#1\strut}}
\newcommand{\CSLRightInline}[1]{\parbox[t]{\linewidth - \csllabelwidth}{\strut#1\strut}}
\newcommand{\CSLIndent}[1]{\hspace{\cslhangindent}#1}

\usepackage{float}
\usepackage{tabularray}
\usepackage[normalem]{ulem}
\usepackage{graphicx}
\UseTblrLibrary{booktabs}
\UseTblrLibrary{rotating}
\UseTblrLibrary{siunitx}
\NewTableCommand{\tinytableDefineColor}[3]{\definecolor{#1}{#2}{#3}}
\newcommand{\tinytableTabularrayUnderline}[1]{\underline{#1}}
\newcommand{\tinytableTabularrayStrikeout}[1]{\sout{#1}}
\usepackage{lastpage}
\usepackage{fancyhdr}
\fancypagestyle{fancy}{
 \fancyhf{} % clear all header and footer fields
 \fancyfoot[C]{Side \thepage\ af \pageref*{LastPage}}
 \renewcommand{\headrulewidth}{0pt}
 \renewcommand{\footrulewidth}{0pt}
}
\fancypagestyle{plain}{
 \fancyhf{}
 \fancyfoot[C]{Side \thepage\ af \pageref*{LastPage}}
 \renewcommand{\headrulewidth}{0pt}
 \renewcommand{\footrulewidth}{0pt}
}
\pagestyle{fancy}

\usepackage{float}
\floatplacement{table}{H}

\usepackage{caption}

\captionsetup{justification=raggedright,singlelinecheck=false}

\usepackage{lipsum}

\usepackage{icomma}
\usepackage{siunitx}
\sisetup{
  add-decimal-zero = false , % default setting, not needed
  output-decimal-marker = {,} ,
  group-separator = \, % default setting, not needed
}
\usepackage{lastpage}
\usepackage{fancyhdr}
\fancypagestyle{fancy}{
 \fancyhf{} % clear all header and footer fields
 \fancyfoot[C]{Side \thepage\ af \pageref*{LastPage}}
 \renewcommand{\headrulewidth}{0pt}
 \renewcommand{\footrulewidth}{0pt}
}
\fancypagestyle{plain}{
 \fancyhf{}
 \fancyfoot[C]{Side \thepage\ af \pageref*{LastPage}}
 \renewcommand{\headrulewidth}{0pt}
 \renewcommand{\footrulewidth}{0pt}
}
\pagestyle{fancy}
\KOMAoption{captions}{tableheading,figureheading}
\makeatletter
\@ifpackageloaded{caption}{}{\usepackage{caption}}
\AtBeginDocument{%
\ifdefined\contentsname
  \renewcommand*\contentsname{Indholdsfortegnelse}
\else
  \newcommand\contentsname{Indholdsfortegnelse}
\fi
\ifdefined\listfigurename
  \renewcommand*\listfigurename{Figuroversigt}
\else
  \newcommand\listfigurename{Figuroversigt}
\fi
\ifdefined\listtablename
  \renewcommand*\listtablename{Tabeloversigt}
\else
  \newcommand\listtablename{Tabeloversigt}
\fi
\ifdefined\figurename
  \renewcommand*\figurename{Figur}
\else
  \newcommand\figurename{Figur}
\fi
\ifdefined\tablename
  \renewcommand*\tablename{Tabel}
\else
  \newcommand\tablename{Tabel}
\fi
}
\@ifpackageloaded{float}{}{\usepackage{float}}
\floatstyle{ruled}
\@ifundefined{c@chapter}{\newfloat{codelisting}{h}{lop}}{\newfloat{codelisting}{h}{lop}[chapter]}
\floatname{codelisting}{Liste}
\newcommand*\listoflistings{\listof{codelisting}{Listeoversigt}}
\makeatother
\makeatletter
\makeatother
\makeatletter
\@ifpackageloaded{caption}{}{\usepackage{caption}}
\@ifpackageloaded{subcaption}{}{\usepackage{subcaption}}
\makeatother

\ifLuaTeX
\usepackage[bidi=basic]{babel}
\else
\usepackage[bidi=default]{babel}
\fi
\babelprovide[main,import]{danish}
\ifPDFTeX
\else
\babelfont{rm}[]{Times New Roman}
\fi
% get rid of language-specific shorthands (see #6817):
\let\LanguageShortHands\languageshorthands
\def\languageshorthands#1{}
\ifLuaTeX
  \usepackage{selnolig}  % disable illegal ligatures
\fi
\usepackage{bookmark}

\IfFileExists{xurl.sty}{\usepackage{xurl}}{} % add URL line breaks if available
\urlstyle{same} % disable monospaced font for URLs
\hypersetup{
  pdftitle={Brown \& Enos i Danmark},
  pdfauthor={Aleksander Bang-Larsen},
  pdflang={da},
  hidelinks,
  pdfcreator={LaTeX via pandoc}}


\title{Brown \& Enos i Danmark}
\author{Aleksander Bang-Larsen}
\date{22. oktober 2024}

\begin{document}
\maketitle


\setstretch{1.25}
\section{Motivation}\label{motivation}

Brown og Enos (\citeproc{ref-brown2021}{2021}) undersøger den lokale
geografiske politiske opdeling (segregering) og hvordan den er
forskellig på tværs af geografiske områder og grader af
befolkningstæthed. Geografisk politisk opdeling beskriver det fænomen,
hvor individer med forskellige politiske orienteringer bosætter sig
adskilt fra hinanden. Dette taler ind i litteraturen om socialisering,
gruppekonflikt og kan skabe problemer for hvordan vi optimalt fordeler
ressourcer i vores samfund (\citeproc{ref-brown2021}{ibid.}).

\textbf{Hvorfor er det vigtigt at undersøge i Danmark?} Det er super
interessant at vide, hvordan lokalområder i Danmark påvirkes af deres
naboer. Det giver også god intuitiv mening at en landmand har nogenlunde
samme holdninger som sin nabolandmand, eller at en
universitetsstuderende kan have lignende holdninger som sin nabo der
også er studerende. Det kan særligt have relevans for forskellene mellem
land og by, men i mine øjne har det også stor indvirkning på, hvordan
lokalvalg (kommunal- og regionsrådsvalg) ender.

Brown og Enos (\citeproc{ref-brown2021}{ibid.}) undersøger de
geografiske forskelle mellem \emph{clustering} af demokrater og
republikanere i USA. På samme måde er det relevant at undersøge hvordan
danske politiske tilhængere placerer sig i \emph{clusters}. Der er i
Danmark rimelige forskelle mellem land- og by, både i politisk og
åndelig forstand, men også rent geografisk og naturmæssigt.

\section{Målingsstrategi}\label{muxe5lingsstrategi}

For at fremlægge hvordan jeg vil måle den geografiske opdeling i Danmark
vil jeg først gennemgå min konceptualisering af selve fænomenet. Herfra
vil jeg teoretisk beskrive, hvordan det kan måles; både i den mest
optimale situation og hvordan jeg realistisk vil måle det. Sidstnævnte
vil drive udfordringsafsnittet nedenfor.

Jeg forstår den geografiske politiske opdeling som den mængde
eksponering hvert individ har med andre individer i sit lokalområde
(eller i sin hverdag) der har både ensrettede og modsatrettede politiske
overbevisninger. Fordelingen af partier beskrives i afsnittet nedenfor.
Udover fordelingen af partier vil jeg i en optimal situation opnå
stemmeadfærdsdata for hvert individ i et givent område (f.eks.
kommunerne Aarhus og Herning). Disse individ-niveau data vil jeg kode
sammen med bopælsoplysninger for hvert individ og gennemgå analysen fra
Brown og Enos (\citeproc{ref-brown2021}{ibid.}). Det er dog urealistisk
at opnå vælgeradfærdsdata for hele kommuner med stor grad sikkerhed i
Danmark, da alle stemmer er anonyme og der introduceres hukommelsesbias
i et survey. Det survey vil udover at være meget omfattende også være
meget dyrt - hvis der skulle indsamles data for hele kommuner. Derfor
vil jeg undersøge stemmefordelingen på valgstedsniveau, hvorfra der
opnås stemmeadfærd på alle opstillede partier i relativt små områder.
Her kan der tilfædigt placeres et punkt for hver stemme på hvert parti i
et given afstemningsområde. Alternativt kan disse punkter placeres i
centroiden for området som beskrevet af Logan
(\citeproc{ref-logan2012}{2012: 5}), hvilket også er den tilgang jeg vil
hælde til. Herfra kan analysen fra Brown og Enos
(\citeproc{ref-brown2021}{2021}) kan gennemføres.

De politiske overbevisninger er i Danmark mere spredte end i USA, hvor
der er \(11 (+4)\) partier i folketinget. Det giver en udfordring i,
hvordan man skal inddele individernes orientering. Jeg ser det mest
fordelagtigt at inddele \(4\) grupperinger; Venstrefløjen,
centrum-venstre, højrefløjen og den yderste højrefløj. Jeg vil gruppere
partierne på en skala som ses i Tabel~\ref{tbl-1}.

\begin{table}

\caption{\label{tbl-1}Partiernes inddeling i fire grupper}

\centering{

\centering
\begin{tblr}[         %% tabularray outer open
]                     %% tabularray outer close
{                     %% tabularray inner open
colspec={Q[]Q[]Q[]Q[]},
}                     %% tabularray inner close
\toprule
Venstrefløjen & Centrum-venstre & Højre & Yderste højrefløj \\ \midrule %% TinyTableHeader
Alternativet & Socialistisk Folkeparti & Venstre                     & Danmarksdemokraterne \\
Frie Grønne  & Socialdemokraterne      & Det Konservative Folkeparti & Dansk Folkeparti     \\
Enhedslisten & Det Radikale Venstre    & Liberal Alliance            & Nye Borgerlige       \\
& Moderaterne             &                             &                      \\
\bottomrule
\end{tblr}

}

\end{table}%

\section{Udfordringer}\label{udfordringer}

Ovenfor har jeg beskrevet en metode til at måle de politiske
tilknytninger for borgere på valgstedsniveau. Det kan være problematisk
hvis man vil måle eksponeringen på lavere niveau. Dog synes det
fordelagtigt at vi kan afgrænse eksponeringen til dette lukkede (og
relativt lille) område. Derudover synes det at være den mest
tilgængelige metode at opnå data der ligner vælgerregistrene der bruges
i Brown og Enos (\citeproc{ref-brown2021}{2021}).

Ved at tilskrive alle punkterne enten tilfældige placeringer eller
afstemningsområdets centroide som placering vil vi indføre støj der i en
optimal verden ikke var til stede. Hvis det var muligt at tilegne sig
vælgerregistre eller udføre stor-skala survey med repræsentation af alle
områder ville disse muligheder være at foretrække.

Min tilgang kan være teoretisk begrænset i det, at der ikke reelt
måles/spørges til vælgerens politiske overbevisning, eller den
overbevisning de udtrykker i deres lokalsamfund, men blot måles
vælgeradfærd tilbage i tiden. Dog ser jeg ikke koblingen fra politisk
overbevisning til vælgeradfærd som værende et stort problem.

\section{Litteratur}\label{litteratur}

\phantomsection\label{refs}
\begin{CSLReferences}{1}{1}
\bibitem[\citeproctext]{ref-brown2021}
Brown, Jacob R. og Ryan D. Enos (2021).
\href{https://doi.org/10.1038/s41562-021-01066-z}{The Measurement of
Partisan Sorting for 180 Million Voters}. \emph{Nature Human Behaviour}
5 (8): 998--1008.

\bibitem[\citeproctext]{ref-logan2012}
Logan, John R. (2012).
\href{https://doi.org/10.1146/annurev-soc-071811-145531}{Making a
{Place} for {Space}: {Spatial Thinking} in {Social Science}}.
\emph{Annual review of sociology} 38.

\end{CSLReferences}




\end{document}
